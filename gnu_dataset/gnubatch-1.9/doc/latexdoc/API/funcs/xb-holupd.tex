\subsection{\funcnameXBholupd{}}

\begin{expara}

int \funcnameXBholupd{}(const int fd,

\ \ \ \ \ \ \ \ \ \ \ \ \ \ const unsigned flags,

\ \ \ \ \ \ \ \ \ \ \ \ \ \ int year,

\ \ \ \ \ \ \ \ \ \ \ \ \ \ unsigned char *bitmap)

\end{expara}

The function \funcXBholupd{} is used to update the holiday file for the specified year.

\exampletext{fd} is a file descriptor which was previously returned by a successful call to \funcXBopen{} or equivalent.

\exampletext{flags} is currently unused but is reserved for future use.

\exampletext{year} is the year for which the holiday file is required. This should be the actual number of the year or an offset
from 1900. For example the year 1994 could be given as 1994 or 94. Note: The offset value should be less than 200.

\exampletext{bitmap} is an array of characters representing the bitmap. Bits are set if the days is a holiday. To test the bitmap use
the following formula:

\begin{expara}

if (bitmap[day {\textgreater}{\textgreater} 3] \& (1
{\textless}{\textless} (day \& 7)))

\ \ \ \ /*day is holiday*/

\end{expara}

\subsubsection{Return values}
The function returns 0 if successful otherwise one of the error codes listed on page \pageref{errorcodes} onwards.

