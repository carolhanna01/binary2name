\subsection{\funcnameXBjobchmod{}}

\begin{expara}

int \funcnameXBjobchmod{}(const int fd,

\ \ \ \ \ \ \ \ \ \ \ \ \ \ \ \ const unsigned flags,

\ \ \ \ \ \ \ \ \ \ \ \ \ \ \ \ const slotno\_t slot,

\ \ \ \ \ \ \ \ \ \ \ \ \ \ \ \ const Btmode *newmode)

\end{expara}

The function \funcXBjobchmod{} is used to change the
permissions of a job.

\exampletext{fd} is a file descriptor which was previously
returned by a successful call to \funcXBopen{} or equivalent.

\exampletext{flags} is zero or
\filename{\constprefix{}FLAG\_IGNORESEQ} to ignore recent changes to
the job list.

\exampletext{slot} is the slot number corresponding to the job
as returned by \funcXBjoblist{} or
\funcXBjobfindslot{}.

\exampletext{newmode} is a pointer to a structure containing
the details of the new mode.

The definition of the job structure is given on page
\pageref{bkm:Jobstructure} onwards.

\subsubsection{Return values}
The function returns 0 if successful otherwise one of the error codes
listed on page \pageref{errorcodes} onwards.

