\subsection{\funcnameXBjobchgrp{}}

\begin{expara}

int \funcnameXBjobchgrp{}(const int fd,

\ \ \ \ \ \ \ \ \ \ \ \ \ \ \ \ const unsigned flags,

\ \ \ \ \ \ \ \ \ \ \ \ \ \ \ \ const slotno\_t slot,

\ \ \ \ \ \ \ \ \ \ \ \ \ \ \ \ const char *newgroup)

\end{expara}

The function \funcXBjobchgrp{} is used to attempt to
change the group ownership of a job.

\exampletext{fd} is a file descriptor which was previously
returned by a successful call to \funcXBopen{} or equivalent.

\exampletext{flags} is zero or
\filename{\constprefix{}FLAG\_IGNORESEQ} to ignore recent changes to
the job list.

\exampletext{slot} is the slot number corresponding to the job
as returned by \funcXBjoblist{} or
\funcXBjobfindslot{}.

\exampletext{newgroup} is a valid group name.

\subsubsection{Return values}
The function returns 0 if successful otherwise one of the error codes
listed on page \pageref{errorcodes} onwards.

