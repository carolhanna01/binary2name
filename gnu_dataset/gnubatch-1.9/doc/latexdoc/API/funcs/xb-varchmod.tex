\subsection{\funcnameXBvarchmod{}}

\begin{expara}

int \funcnameXBvarchmod{}(const int fd,

\ \ \ \ \ \ \ \ \ \ \ \ \ \ \ \ const unsigned flags,

\ \ \ \ \ \ \ \ \ \ \ \ \ \ \ \ const slotno\_t slot,

\ \ \ \ \ \ \ \ \ \ \ \ \ \ \ \ const Btmode *newmode)

\end{expara}

The function \funcXBvarchmod{} is used to change the
permissions associated with a variable.

\exampletext{fd} is a file descriptor which was previously
returned by a successful call to \funcXBopen{} or equivalent.

\exampletext{flags} is 0 or
\filename{\constprefix{}FLAG\_IGNORESEQ} to ignore recent changes to
the variable list if possible.

\exampletext{slot} is the slot number corresponding to the
variable as returned by \funcXBvarlist{} or
\funcXBvarfindslot{}.

\exampletext{newmode} is a pointer to the structure which
contains all the new mode details. The definition of the variable
structure is given on page \pageref{bkm:Varstructure} onwards.

\subsubsection{Return values}
The function returns 0 if successful otherwise one of the error codes
listed on page \pageref{errorcodes} onwards.

