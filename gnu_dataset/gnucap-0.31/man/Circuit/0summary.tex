%$Id: 0summary.tex,v 21.13 2002/03/25 05:37:03 al Exp $
% man circuit summary .
% Copyright (C) 2001 Albert Davis
% Author: Albert Davis <aldavis@ieee.org>
%
% This file is part of "Gnucap", the Gnu Circuit Analysis Package
%
% This program is free software; you can redistribute it and/or modify
% it under the terms of the GNU General Public License as published by
% the Free Software Foundation; either version 2, or (at your option)
% any later version.
%
% This program is distributed in the hope that it will be useful,
% but WITHOUT ANY WARRANTY; without even the implied warranty of
% MERCHANTABILITY or FITNESS FOR A PARTICULAR PURPOSE.  See the
% GNU General Public License for more details.
%
% You should have received a copy of the GNU General Public License
% along with this program; if not, write to the Free Software
% Foundation, Inc., 59 Temple Place - Suite 330, Boston, MA
% 02111-1307, USA.
%------------------------------------------------------------------------
\section{Summary}

 To describe a circuit, you must provide a `netlist'.  The netlist is simply
a list of the components with their connections and values.  The format is
essentially the same as the standard SPICE format.

Before doing this, number the nodes on your schematic.  (A node is a place
where parts connect together.)  Then, each part gets a line in the netlist
(circuit description).  In its simplest form, which you will use most of the
time, it is just the type, such as `R' for resistor, or a label, like `R47',
followed by the two nodes it connects to, then its value.

Example:  `{\tt R29 6 8 22k}' is a 22k resistor between nodes 6 and 8.

\index{ground node}
\index{common node}
\index{voltage reference}
 Node 0 is used as a reference for all calculations and is assumed to have a
voltage of zero.  (This is the ground, earth or common node.)  Nodes must be
nonnegative integers, but need not be numbered sequentially.

\index{open circuit error}
 There should be a DC path through the circuit to node 0 from every node
that is actually used.  The circuit cannot contain a cutset of current
sources and/or capacitors.  If either of these cases occurs, it will be
discovered during analysis.  The program will attempt to correct the error,
issue an `open circuit' error message and continue.  This is rarely a
problem with real circuits.  Most circuits have such a path, however
indirect.

Semiconductor devices require both a device statement, and a {\tt .model}
statement (or ``card'').  The device statement, described in the Circuit
description chapter, defines individual devices as variations from a
prototype, as is required for different devices on the same substrate.  The
model statement, described in this chapter, defines process dependent
parameters, which usually apply to all devices on a substrate.

The {\tt .model} card syntax is:
\begin{verse}
{\tt .model} {\it mname type} \{{\it args}\}
\end{verse}

{\it Mname} is the model name, which elements will use to refer to this
model.  {\it Type} is one of several types of built-in models.  {\it Args}
is a list of the parameters, of the form {\it name}{\tt =}{\it value}.

\begin{description}

\item[{\tt D}] Diode model

\item[{\tt NMOS}] N-channel MOSFET model

\item[{\tt PMOS}] P-channel MOSFET model

\item[{\tt LOGIC}] Logic family description

\item[{\tt SW}] Voltage controlled switch

\item[{\tt CSW}] Current controlled switch

\item[{\tt C}] Semiconductor capacitor

\item[{\tt R}] Semiconductor resistor

\item[{\tt TABLE}] y/x table of values

\end{description}
%------------------------------------------------------------------------
%------------------------------------------------------------------------
