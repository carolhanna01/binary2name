\documentstyle[12pt,latexinfo,tabular]{report}
\pagestyle{headings}

\begin{document}

\newindex{cp}
\newindex{vr}
\newindex{fn}
\newindex{tp}
\newindex{pg}
\newindex{ky}

\comment %**start of header (This is for running Texinfo on a region.)
\setfilename{dominion.info}
\markboth{Dominion Manual}{Dominion Manual}
\comment %**end of header (This is for running Texinfo on a region.)

\begin{tex}
\pagestyle{empty}
\title{Dominion Manual}
\author{Mark Galassi and Michael Fischer}
\date{\today}

\maketitle
\pagestyle{headings}
\pagenumbering{roman}
\tableofcontents

\comment  The following two commands start the copyright page.
\clearpage
\comment \vskip 0pt plus 1filll

Copyright \copyright{} 1994, 1990 Free Software Foundation, Inc.

Permission is granted to make and distribute verbatim copies of this manual
provided the copyright notice and this permission notice are preserved on
all copies.

Permission is granted to copy and distribute modified versions of this
manual under the conditions for verbatim copying, provided also that the
GNU Copyright statement is available to the distributee, and provided that
the entire resulting derived work is distributed under the terms of a
permission notice identical to this one.

Permission is granted to copy and distribute translations of this manual
into another language, under the above conditions for modified versions.

\clearpage
\end{tex}

\begin{ifinfo}
This file documents dominion.

Copyright @copyright{} 1990, 1994 Free Software Foundation, Inc.
Authored by the dominion project (see Authors section of this manual).

Permission is granted to make and distribute verbatim copies of this manual
provided the copyright notice and this permission notice are preserved on
all copies.
\begin{ignore}
Permission is granted to process this file through Tex and print the
results, provided the printed document carries copying permission notice
identical to this one except for the removal of this paragraph (this
paragraph not being relevant to the printed manual).
\end{ignore}
Permission is granted to copy and distribute modified versions of this
manual under the conditions for verbatim copying, provided also that the
GNU Copyright statement is available to the distributee, and provided that
the entire resulting derived work is distributed under the terms of a
permission notice identical to this one.

Permission is granted to copy and distribute translations of this manual
into another language, under the above conditions for modified versions.
\end{ifinfo}

\clearpage
\pagenumbering{arabic}

\chapter{Overview}

{\em Dominion} is a multi-player world simulation game.  Each player
is the leader of a nation, and makes decisions for that nation.  The
decisions are political, military, diplomatic and economic, and all
are extremely important for the well-being of a nation.  Some nations
can be played by the computer.  These nations are called {\em CNs}
(Non Player Countries).  They play a challenging game, and are quite
useful if few human players are available.

Dominion has features from fantasy role-playing games, educational
games, and war games. A player needs to develop a character as the
leader of a nation, keep a healthy economy, and develop a strong
military force using magic or technology.

The dominion world is broken up into small areas called {\em sectors}.
Each nation tries to occupy sectors and employ people in them.
The nation's leader decides what the purpose of each sector should be,
and what the people in the sector will produce.

All nation leaders possess a magical palantir which allows them to
communicate with each other.  Nations may send mail to each other,
or post publicly readable articles to various newsgroups.  Nations
also receive mail from the dominion game itself informing them of
events in their nation and the world.

Each leader must try to obtain the resources needed to run his nation,
and use them wisely.  The main resources are {\em money}, {\em metal},
and {\em jewels}. The unit of currency in Dominion is {\em sheckles}
(abbreviated sk.)  These resources can be spent for various things,
such as building cities, drafting armies, or hiring mages to perform
magic.  In addition, resources can be invested in technological or
magical research. This research will yield new technology which will
improve your nation, and new magic spells and spirits will become
available to your mages.

Most of the moves you make are not resolved until the end of a turn,
when the {\em update} is run.  This update will incorporate your
changes into the world database, then it will update your economy,
handle migration of people, resolve battles and conquest of land, and
a few other things.  You will receive mail informing you of what
happened to your nation during the update. The time elapsed between
updates is called a {\em thon}.

\chapter{Getting and installing Dominion}

Dominion is available by anonymous ftp from {\tt
insti.physics.sunysb.edu}, in the directory {\tt /pub}.  As I write,
the latest version there is {\tt dominion-2.4.tar.gz}. The URL for
that is {\tt ftp://insti.physics.sunysb.edu/pub/} and you should check
which the most recent version is in that directory.

You might also want to get the latexinfo package, in case you want to
do things with the documentation (though we provide the .info files).
We provide latexinfo on the same anonymous ftp site: {\tt
ftp://insti.physics.sunysb.edu/pub/latexinfo.tar.gz}.

Dominion now follows the GNU coding standards, which means that the
installation is quite straightforward; you unbundle the software with
\begin{verbatim}
gzcat dominion-version.tar.gz | tar xvf -
\end{verbatim}

Then you type
\begin{verbatim}
./configure --prefix=your-prefix-dir
\end{verbatim}

After which you can type
\begin{verbatim}
make install
\end{verbatim}

After which you are ready to make a world with ({\tt dom_make}), add the
computer nations with ({\tt dom_add -f cns}), add human-played nations
({\tt dom_add}) and do updates ({\tt dom_udpate}).

The steps to make a new world can be abbreviated by running

\begin{verbatim}
make new-world
\end{verbatim}	

The gamemaster manual has more information on how the gamemaster can
run the game.


\chapter{Getting started}

\section{Choosing your nation}

Your nation is added to the game by the {\em Gamemaster}, or someone
trusted by the Gamemaster.  You get to choose the {\em name} of your
nation, your {\em leader's name}, your {\em nation mark}, the {\em
race} of your people, and the {\em magical order} for your nation.

Your nation mark is a single character which is used to denote your
territories on a map.  Most players try to use the first letter of
their nation name for this.  Each letter may only be used once
however, so you should have a few choices ready in case your preferred
one is already taken.  In some games the Gamemaster may assign nation
marks to nations.

The choice of race plays a key role in dominion. Each race has certain
characteristics (dwarves are better miners; elves are more
intelligent, orcs reproduce like crazy, etc.) which will affect many
different aspects of your game. Also, the choice of race largely
determines the role you will play in the game. The different races
available in Dominion are described in section \ref{sec-races}.

Your magical order determines what spirits you can summon, and which
spells you may cast.  If you are planning on using much magic in
dominion, this choice should be made very carefully.  The different
magic orders are described in Section \ref{sec-magic}.

\section{The screen display}

When you first look at your nation in Dominion, you will see a screen
divided into four main sections.  The largest area of the screen is
your nation's sector map.  Each sector is represented on the
screen by a letter or symbol, with your sectors highlighted.  Your
terminal's cursor should be on the ``\key{C}'' in the center of the screen.

Your sectors are described by a letter, which tells you the
designation of the sector.  This designation determines the purpose of
the sector.  The designations you might see at the beginning of a game
are ``\key{C}'', which is your nation's capital, ``\key{m}'', a metal
mine, ``\key{j}'', a jewel mine, or ``\key{f}'', a farm.  You can
change these designations, and set the designation of new sectors you
occupy.  This will typically cost money, and sometimes metal and
jewels.

The box in the lower right corner of the screen is the {\em sector
window}.  It gives a brief description of the sector that the
cursor is currently on.  The window tells you the sector's location,
owner, designation, and other important information.  All coordinates
in Dominion are displayed relative to your capital.

Above the sector window is a list of armies.  This list gives brief
information about each army in the sector.  It is used to select specific
armies to move or manipulate.

The bottom two lines of the screen are the status lines.  These lines
are used by dominion to display information or status messages,
or to get input from the player.

\section{Moving around the map}

You can move the cursor to another sector on the map using the [h],
[j], [k], and [l] keys to move left, down, up, and right, and the keys
[y], [u], [b], and [n] to move diagonally.  This is similar to the
cursor movements in some editors (like {\em vi}), and some UNIX games
(such as {\em rogue}, {\em larn}, {\em nethack}, and {\em conquer}).
For large movements across the map, you can use the {\em upper-case}
letters [H], [J], [K], and [L].  These jump 8 sectors in the specified
direction .\footnote{Throughout this manual, letters inside brackets
indicate pressing the specified key.  For example, {\em the [Z]oom
window} means typing a capital Z will bring you to the zoom window.}

Alternatively, you may use the numeric keypad, in which case the
number keys move you in the same direction in which they point on the
keypad.  Both the ordinary keys and the numeric keypad can be used to
browse the map and to move armies.  The following diagram shows you
the directions and keys you can use.

\comment hmm, think about keeping this on the same page.
\comment \newpage
\begin{verbatim}
                               NORTH

                                (K)
                          y,7   k,8   u,9
                             \   |   /
                               \ | /
          WEST      (H) h,4 ---- 0 ---- l,6 (L)       EAST
                               / | \
                             /   |   \
                          b,1   j,2   n,3
                                (J)

                               SOUTH
\end{verbatim}

\section{Moving your armies}
Armies are manipulated with the [a]rmy menu.  Here are {\em some} of
the army commands.  A complete list comes later.

\begin{itemize}

\item
\strong{[l]ist}: This will give you a list of army and navy types
available for drafting. At the start, you should see {\em Cavemen} and
{\em Caravans} in the list. As your technology improves, more army and
navy types will become available.
\item
\strong{[n]ext army} and \strong{[p]revious army}: These pick the next
and previous armies in the current sector, skipping armies that do not
belong to you.  The currently selected army will be highlighted in the
army list.
\item
\strong{[m]ove}: After you have selected an army you want to move type
[m] and move the cursor to where you want the army to be.
\strong{Warning}: watch your army's {\em move points} and the
{\em move cost} of each sector you cross: if you run out of move points
your army might get stuck where you don't want it!
\item
\strong{[s]tatus}:  This lets you change your army's status.  The status
affects what your army will do in various situations.  The most basic
statuses are:

\begin{itemize}
\item
\strong{[a]ttack}: When in this mode your army will attack an enemy
nation's army in the same sector.  This will happen only if you are at
{\em war} or {\em jihad} with the other nation.
\item
\strong{[d]efend}: An army in defend status can not take a sector nor
will it attack another army.  If attacked it will defend itself.
\item
\strong{[o]ccupy}: If you want to take a sector that is un-owned or owned
by an enemy, you move an army of 100 or more soldiers to that sector
and set its status to occupy.
\end{itemize}

\item
\strong{[d]raft}: To draft an army or navy you must be in a city or
capital that you own.  Drafting an army will cost metal and money, so
be careful and watch how much you spend.  You will be given the list
of available army types, (see [l]ist), and you must type the
abbreviation for that army type.

\end{itemize}

\section{Taking sectors}
At the beginning of the game, you should try to take many sectors:
this gives a safety buffer around your capital, and allows you to look
for resources in the new occupied lands.  You take sectors by moving
an army {\em of at least 100 soldiers} to that sector and setting it
on occupy mode.  If the sector belongs to another nation, you will
have to declare war to take the sector from them.  You declare
war using the [r]eports menu, and choosing [d]iplomacy.

Once you take a sector (it will become yours after the update), you
can redesignate that sector so it produces what you want.  For
example, redesignating to a farm will make that sector produce food.
Sector designations are described in great detail later, but you
should know now that to change a designation you use the [Z]oom key to
focus on the specific sector, and then change the designation with the
[r]edesignate key.  You will be given a menu of possible designations.

To help you choose a designation for the sectors you take, the sector
window shows you the {\em soil}, {\em metal}, and
{\em jewels} in that sector.  For example, if the sector has a
high metal yield, you might want to make a {\em metal mine} out of
it.

Below is an example of a sector window that shows a sector with jewels
5, metal 0 and soil 6.  The sector belongs to nation Khazad Dum, has
coordinates (2, 2) relative to the current player, has 452 inhabitants
which are of race (D), Dwarves.

\comment Hmm:  think about putting it on the same page or a new page
\comment \newpage
\begin{verbatim}
                                +----------------------+
                                |(2,2)                 |
                                |Khazad Dum-jwl. mine  |
                                |Brush Plateau         |
                                |452 people (D)        |
                                |metal 0     jewels 5  |
                                | soil 6   movecost 1  |
                                +----------------------+
\end{verbatim}

\section{Setting up your budget}
It is important that you set up your budget properly. Do this by
choosing the [r]eports menu, and then the [b]udget report.  Here
you choose your tax rate, and the amount of money, metal, and
jewels you wish to invest in various types of research.  A note
about taxes: if your taxes are too high, your production of food,
metal and jewels will decrease.

There are default values set for investment in magic and technology,
and there is a default tax rate.  This is intended to guide new
players through their first moves.

The {\em Breakdown of Expenditures} in the report shows how much of
each thon's {\em revenue} you are investing in various things.  You
can also spend a fraction of your {\em reserves} on various types of
research, by selecting the [s]torage option. This amount will be reset
to 0 after each update. The report also shows you how much of each
resource you are spending, and your predicted resources for next thon.

The budget report is explained in more detail in Section
\ref{sec-budget-report}.

\section{Commands}
Below is a brief list of dominion commands.  It is actually a copy of
the reference card available as on-line help in dominion (you access
this with the [?] key followed by [r]eference).
\comment \newpage
\begin{verbatim}
                       Dominion QUICK REFERENCE CARD

Display:
    [d]isplay options     [F] dump map to file   [w]indow manipulation
    [^L] redraw screen    [p] jump to a point    [P] jump to your capital
Administration:
    [r]eports             [a]rmies               [Z]oom on sector
    [W]izardry            [t]ransportation       [C]onstruct
Miscellaneous:
    [Q]uit (or [q]uit)    [m]ail                 [N]ews
    [O]ptions

\end{verbatim}

\chapter{The Dominion world}

\section{Races}
\label{sec-races}
There are many races available to players in dominion, and each Game
Master can add new races by modifying the race descriptor file.
This should not be done lightly however, as the parameters describing
each new race must be carefully tuned to preserve game balance.
The races currently available are Elf, Human, Dwarf, Orc, Merfolk,
Icefolk, Hobbit, Gnome, Harpy, Ogre, Walrus, Algae, and Squid.

\subsection{Race parameters}

The parameters describing your race are strength, reproduction rate,
mortality rate, intelligence, speed, stealth, preferred altitude,
vegetation, and temperature, and aptitude for magic, farming and mining.

In addition to these parameters, some races have certain special army
types available to them.  For example, Harpies can draft armies of
type ``Harpy'' which can fly.  The army types table (Table
\ref{tab-army-types}) lists the race specific armies and their
characteristics.

Some of the races (Merfolk, Walrus, Algae, and Squid) live under
water.  The game is almost symmetrical for races that live above and
below water. Races that live above and below water can interact (and
fight) in several ways, as described below.

Each of the above parameters affect the races in the following ways:

\begin{itemize}
\item
\strong{Strength}: affects your combat bonus.
\item
\strong{Reproduction}: the rate at which people are born in your nation.
\item
\strong{Mortality}: every year this percentage of your population dies.
\item
\strong{Intelligence}: affects your acquisition of technological skill,
and also helps in combat.
\item
\strong{Speed}: affects your armies' move points.
\item
\strong{Stealth}: affects your spy and secrecy skills.
\item
\strong{Preferred altitude, vegetation, and temperature}: these affect
migration of people, and the move cost of sectors.  For example, if
your race prefers high altitude, mountains will have less move cost.
\item
\strong{Magic aptitude}: this affects how quickly you learn the spells for
your chosen magical order, and how many spell points you gain.
\item
\strong{Farming}: your farms produce this percentage more than their basic
productivity.  This will increase during the game, as you acquire new
technology.
\item
\strong{Mining}: your metal and jewel mines produce this percentage more than
their basic productivity.  This will increase during the game as you
acquire new technology.
\end{itemize}

Table \ref{tab-races} below shows the parameters for each race available
in dominion.

\begin{same}
\begin{table}[hbpt]
\caption{Race Types}
\label{tab-races}
\begin{tabular}{ || l | r | r | r | r | r | r | r | r | r | r | r | r || }
\hline
Race    &Str&Rep&Mort&Intel&Spd&Stl&Alt&Veg&Temp&Mag&Farm&Mine\\
\hline
Human   & 80& 11&   8&   50& 65&  4&  2&  3&   7& 30&   0&   0\\
Elf     & 70&  8&   5&   70& 80&  8&  2&  5&   7& 55&  10& -15\\
Orc     & 50& 15&  10&   20& 40&  3&  4&  4&   4& 35&   0&   5\\
Dwarf   & 95&  9&   6&   60& 40&  2&  5&  3&   6& 30&  -5&  20\\
Hobbit  & 15& 10&   7&   45& 50&  9&  3&  4&   7& 20&   5&   5\\
Merfolk & 30&  7&   4&   75& 80&  7& -2& -1&   4& 55&  50& -10\\
Icefolk & 90&  9&   7&   50& 70&  4&  3&  0&  10& 30&  50&   0\\
Gnome   & 75& 10&   8&   95& 40&  9&  4&  3&   5& 10&   0&  10\\
Harpy   & 40& 12&  10&   25& 60&  5&  5&  4&   6& 30&   0&  -5\\
Ogre    & 95&  5&   3&   75& 60&  5&  4&  3&   6& 50&   5&   5\\
Walrus  & 95&  9&   7&   30& 80&  3& -1& -1&   7& 55&  50&   5\\
Algae   & 15& 11&   6&   50& 60&  7& -2& -1&   3& 45&  80&   0\\
Squid   & 40&  9&   8&   25&110&  8& -2& -1&   5& 40&  50&  15\\
\hline
\end{tabular}
\end{table}
\end{same}

\subsection{Land-water interaction}
For the most part, land races operate on land and water races in the
water.  However, each can extend its influence to the other side of
the sea level in various ways.  Some armies have special flags which
indicate whether they travel in land, water, or both.

Ships have the {\em inverse altitude} flag, \key{I}, which means that
when owned by a land race, ships travel on water.  For a water race,
ships travel on land.

If a land nation would like to occupy sectors in the water it can do
so using armies which have the {\em water} flag, \key{W}, such as {\em
Swimmers} or {\em Scuba_divers}. Similarly, a water race can occupy
land sectors using armies such as {\em Walkers} which have the {\em
land} flag, \key{L}.  Notice that any other army or spirit with the
\key{W} and \key{L} flags will work fine.  An army with both the \key{L}
and \key{W} flags, such as the magical {\em Sea Serpent} can travel
anywhere.

Some of the army types mentioned above have the {\em front line} flag,
\key{f}. This means that they can be unloaded from a ship onto an
unowned sector.  Armies without this flag can only be unloaded onto
land already owned by you.

Once the sector is occupied you cannot move people into it or they
will drown (or suffocate if you are a water nation).  You can, at
great expense, build a {\em bubble} over the sector or cast a {\em
change altitude} spell on it which will allow your people to move into
it, even if it is a water (or land) sector.  This provides a means for
colonizing the oceans (continents).

\section{Technology}
\label{sec-technology}
A nation starts with very low skills in mining and farming. These
skills can be developed by investing money or metal in technological
research and development (R\&D).  This investment is made in the
budget report.  Increases in technology depends on how much metal and
money you invest each thon.  Metal increases technology proportional
to the {\em 3/4 power} of the amount invested; money increases
technology proportional to the {\em square root} of the amount of
money invested.  This means that doubling your metal investment will
promote your technology research much more than doubling your money
will.

With each new technology you develop, you can gain certain things.
For example, {\em fire} technology increases your nation's mining
ability, decreases your people's mortality rate, and increases their
farming ability.  With new technology you can also gain the ability to
draft new types of armies.

\section{Magic}
\label{sec-magic}
Your nation is initiated to one of the magic orders available.  Each
order is characterized by the set of {\em spells} known to that
order, and a set of {\em spirits} that can be summoned by mages of
that order.  When you are initiated as a national leader you know only
a little of the magic of that order, but you can increase your
knowledge by investing money and jewels into magical research.  As you
invest more and more, your magic skill will increase, and you will
learn more advanced spells, and how to summon more powerful spirits.
This investment is made with the budget report.

Your magic skill increase is proportional to the jewels invested in
magic research, and to the square root of the money you invest.  As
with technology, this means that investing jewels will improve your
magic skill more than investing money will.

In addition, some magical orders bring a set of characteristics to
its initiates.  These are described below in the descriptions of the
individual magic orders.

\subsection{Spell points}

To {\em use} magic, that is, to cast spells and summon spirits, you
must acquire {\em spell points}. You get spell points in proportion to
the amount of jewels you invest in magical R\&D, but {\em not}
from your money investment.  Your budget report indicates how many
spell points you will earn from your jewel investment.

\strong{Note:} spell points do {\em not} accumulate: if you don't use
your spell points, two thirds of them will be lost the next thon. It
is advisable to use all the spell points made in a thon, or to wait
and save up jewels and then invest them all at once.

\subsection{Mages}

Mages are necessary for casting spells and summoning spirits.  They
must be initiated inside a capital, city, temple or university.  They
cost 5000 jewels to initiate, and 1000 jewels in maintainance each
thon. Mages are moved around as if they were an army, and have twice
the nation's basic move rate. Some spirits have the {\em wizardry} flag,
and they behave like mages.

You need to bring a mage to a certain sector to cast a spell in it or
to summon a spirit there, so you should make sure you initiate some
mages to work for your nation once you are ready to use magic.  They
cost a lot to initiate and maintain, but are worthwhile.

If a mage is on a sector where there is a battle, he will try to stay
out of the battle.  If 85% of the mage's accompanying force is killed,
however, he will be killed also.

See section \ref{sec-wizardry} on the [W]izardry command for details
on using magic.

\subsection{Magic Orders}

The basic magic orders in dominion are:

\begin{itemize}

\item
\strong{Aule}.  Aule is the god of the earth, and is interested
in all that happens in the depths of the earth.  He protects miners
and workers of metal and stone, and gives them an extra 10% mining
skill.

\item
\strong{Avian}.  This order allows you to summon spirits 
related to the air.

\item
\strong{Chess}.  This order allows you to summon spirits
similar to those on the chessboard in their movements and strength.

\item
\strong{Demonology}. This is concerned with the conjuring of demons.

\item
\strong{Diana}.  This order is concerned with animals and hunting.

\item
\strong{Inferno}.  This order is concerned with power through
fire.

\item
\strong{Monsters}.  This order has available many monstrous
creatures.

\item
\strong{Necromancy}.  Necromancy is concerned with the
invocation of dead and undead spirits.  Nations of this order start
with their mortality rate increased by 2%.

\item
\strong{Neptune}.  Neptune is the god of the oceans, and his
order is concerned with the waters.

\item
\strong{Time}.  This order steps back to the age of the
dinosaurs.

\item
\strong{Unity}.  This order has spirits that combine different
creatures in one body.

\item
\strong{Yavanna}.  Yavanna is the godess of plants, and
everything that grows and is fertile is under her protection.  Yavanna
gives her initiates an extra 50% farming skill.

\item
\strong{Insects}.  This order has spirits from the insect
world.  The order gives a nation 1% extra reproduction.

\end{itemize}

The Game Master can add other magical orders to the game by inserting
a list of spirits for that order into a file.

\subsection{Magic Spells}

Here is a description of several spells available in dominion.  Keep
in mind that some spells may be available to several orders, and their
cost and duration will be different for the different orders.

Both the cost and the duration of the spell are indicated when you
list your available spells.  The cost is in spell points.  If the
spell is applied to a sector, then that cost is all you spend.  A
spell applied to an army will usually set a special flag for that
army, and will cost the given amount per 100 men.  Army flags are
described in the section on armies.

Notice that some armies and spirits come into the world with some of
these magical properties already set, so you do not need to set them
with a spell.

\begin{itemize}
\item
\strong{caltitude}
This spell allows you to raise or lower the altitude of a sector.
This means that you could plunge it into the sea, or place it on a
mountain peak.
\item
\strong{cfertility}
This spell allows you to raise or lower the soil fertility of a sector.
\item
\strong{cmetal}
This spell allows you to raise or lower the metal productivity of a
sector.
\item
\strong{cjewels}
This spell allows you to raise or lower the jewel productivity of a
sector.
\item
\strong{fireburst}
This spell devastates the chosen sector, redesignates it to
{\em none}, and sets the soil productivity to zero.
\item
\strong{inferno}
This spell will kill all population in the sector, and make it
completely impenetrable to any army for its duration.  You can only
cast {\em inferno} on your own sectors.  It is a very effective
way to block enemy armies.
\item
\strong{hide_sector}
This spell completely hides the current sector from any other nation.
The sector will just appear as a blank spot on the map.
\item
\strong{hide_army}
This spell sets the {\em hidden} flag on an army.
\item
\strong{fly_army}
This sets the {\em flying} flag on an army.
\item
\strong{vampire_army}
This sets the {\em vampire} flag on an army.
\item
\strong{burrow_army}
This sets the {\em underground} flag on an army.
\item
\strong{water_walk}
This sets the {\em water} or {\em land} flag on an army.  It allows
armies which normally walk on land to also travel in water, and vice versa.
\item
\strong{mag_bonus}
This gives magical enhancement to an army.  The army will fight with an
extra 30% bonus.
\item
\strong{merge}
This spell allows you to merge civilians into spirits (up to twice the
spirits' basic strength), or to take spirits and merge them into the
civilian population of a sector.
\item
\strong{haste_army}
This spell adds the army's basic move rate to its current move points.
This has the effect of doubling the move points of an army in its
normal state.  
\end{itemize}

\subsection{Spirits}

Spirits are like armies, in that they can fight, and they can occupy
sectors (if they are big enough), and their status and movement is
manipulated with the [a]rmy command.

They are also different in many ways.  To obtain spirits, your mage
summons them with spell points.  They are maintained by spell points
each thon (typically 1/3 of the spell points that were needed to
summon them in the first place).

If you don't have enough spell points to support your spirits some of
them will become uncontrolled (see flags) so that you have enough
spell points to control the rest.  Uncontrolled spirits cost no
maintanence, but they will be disbanded over the next update if there
is not enough maintainance for all of the spirits the nation has.

The spirit types for different magic orders are listed in Section
\ref{app-spirits}.

\section{Designations}
If you own a sector, you can {\em redesignate} it.  This specifies
what function that sector has.  When you first occupy an unowned
sector, its designation will be \key{x} (none).  It costs a certain
amount of money, and possibly metal or jewels, to redesignate a
sector.

Each type of sector can employ a different number of people.  For
example, a city can employ several thousand people, whereas a farm can
only properly employ a few hundred people.  The {\em basic maximum}
number of people that a sector can employ is listed in table
\ref{tab-designations} below.  This value is then modified by how
much your race tends to crowd.  If you are an orc, for example, more
people can be crammed into a single sector. The formula to account
for crowding is:
\begin{ifinfo}
	max_employed = sector_max * sqrt(reproduction / 10).
\end{ifinfo}
\begin{tex}
\[ max\_employed = sector\_max \times \sqrt{\frac{reproduction}{10}}. \]
\end{tex}

The more people you have in a sector, the more tax revenue it supplies,
and the more of the sector's product is produced.  A metal mine with
200 people working will produce twice as much metal and tax as a mine with
100 people.  If there are more people than the maximum employable, the
rest will be unemployed, and will produce nothing.  However, they will
of course continue to consume food, so it is a good idea to try to limit
the unemployment in your nation.

The possible designations (together with the characters that are
displayed on the map) are:
\begin{itemize}
\item
\strong{Farm (\kbd{f})} - produces food.  Without farmers producing food,
your people will starve to death.
\item
\strong{Metal mine (\kbd{m})} - supplies your country with metal,
proportional to how good the metal mine is and your mining ability.
\item
\strong{Jewel mine (\kbd{j})} - supplies your country with jewels,
proportional to how good the jewel mine is and your mining ability.
\item
\strong{City (\kbd{c})} - generates a lot of revenue in the form of taxes.
Cities are also the places in which you can draft armies and build
ships.  Cities contain temples, so mages can be initiated in them, and
cities are places of trade, so caravans can drop their goods in
cities.
\item
\strong{Town (\kbd{T})} - a small city in which you can carry out trade,
but not drafting or magic.
\item
\strong{Capital (\kbd{C})} - a city, but the administrative bureaucracy
of your nation is based in your capital, so if your capital is sacked
(occupied by an enemy nation), many of your nation's riches will be
taken.
\item
\strong{University (\kbd{u})} - this is a school of higher education.  Your
country's intelligence can be increased if you put a lot of people in
universities.  Universities are costly to maintain.
\item
\strong{Temple (\kbd{+})} - a place of worship.  Mages can be initiated in
temples.  Also, the fraction of your people in temples increases your
magic skill.
\item
\strong{Fort (\kbd{!})} - forts give bonus to armies camped there (3/turn).
\item
\strong{Hospital (\kbd{h})} - hospitals affect birth and death rates in your
nation. (not yet) Hospitals have maintainance costs each turn.
\end{itemize}

Table \ref{tab-designations}, the {\em Designation table}, describes
the properties of various designations: what they cost, how much
revenue they produce per capita, how much money they cost to maintain,
the minimum employment (not used in all cases), and how many people
can be employed in that sector.

\begin{same}
\begin{table}[hbpt]
\caption{Designation table}
\label{tab-designations}
\begin{tabular}{ || l | l | l | l | l | l | l || }
\hline
Designation &mark&desig& revenue & maint. & min  &  max    \\
            &    & cost & per cap.&per turn&people&employed \\
\hline
None        &  x & 1000 &   30    &    0   &    0 &    7 \\
Farm        &  f & 5000 &  100    &    0   &   10 &  500 \\
Metal mine  &  m &10000 &  100    &    0   &   10 &  800 \\
Jewel mine  &  j &10000 &  100    &    0   &   10 &  800 \\
City        &  c &30000 &  200    &    0   &  300 & 5000 \\
Town        &  T & 8000 &  150    &    0   &   10 &  400 \\
Capital     &  C &50000 &  300    &    0   &  500 & 7000 \\
University  &  u &10000 &   30    & 2000   &  200 & 1000 \\
Temple      &  + & 5000 &    0    &    0   &  200 & 1000 \\
Fort        &  ! &10000 &   50    &    0   &   10 &  200 \\
Hospital    &  h &10000 &  100    & 4000   &   10 &  300 \\
\hline
\end{tabular}
\end{table}
\end{same}

\section{Economy and natural resources}

The pillars of your economy are money and natural resources (soil
fertility, metal and jewels).  How you procure these and use them
plays a major role in dominion.

\subsection{Money}
You get money by levying taxes.  You set the tax rate in the budget
screen.  Set the tax rate wisely to encourage entrepreneurship in your
nation. Production of food, metal and jewels decrease in proportion to
your tax rate.  Money is spent for redesignating sectors, drafting and
maintaining armies, research and development, supporting universities
and hospitals, and many other things.

\subsubsection{Debt}

[NOTE: the section on bonds described here is not yet implemented;
for now a debt simply means negative money]

If your nation's money balance goes negative, you will be forced to
issue bonds to your population to finance the debt (this happens
automatically over the update, so there is no way you can plummet into
a negative balance).  You must then pay an interest on these bonds.

At any time you can also negotiate for other nations to purchase your
bonds.  Interest rate on domestic bonds is fixed (15%/thon), but you
can negotiate the price for bonds issued to other countries.

If you want to finance a big war, and need lots of cash fast, the best
way to go is probably to issue a lot of bonds to other countries.

The bonds your nation issues must always be backed by your
reserve of jewels.

\subsection{Soil fertility}
Your nation must produce the food necessary to feed its people and
soldiers. This is done by designating certain sectors to be farms.
These farms will be more productive if they are on sectors with a
better soil parameter.

Your farming skill also determines how productive your farms will be.
You can increase your farming skill with research in technology, because
your nation will discover better tools and methods for farming.

If your food production is insufficient, your reserves will be used.
If those are not enough, you had better arrange to purchase some, or
that part of your civilians who did not get enough food will starve.
Once all your civilians have starved, your armies will start to
starve. It is also important to know that soldiers eat slightly more
food each thon than people.

\subsection{Metal}
Metal represents all metals and materials used for practical purposes,
such as construction and armaments.  It is found in metal mines.  The
production of each mine is greater if the sector has a higher metal
parameter, and is also increased by your nation's mining skill.  Your
mining skill can increase if you invest in technology, because your
nation will discover better tools for mining and prospecting.

You can spend your metal in several ways, including drafting armies,
technological R\&D, and constructing fortification and roads.

\subsection{Jewels}
In dominion, Jewels represent all kinds of rare resources, such as jewels,
gold, silver, platinum, pearls, and so on.  Jewels are found in jewel
mines.  How many jewels you produce in a mine depends on the jewels
parameter of that sector, and also on your mining ability (see section
on metal).

Jewels are very important, because they can be invested in magical
research, and are used to get spell points.  In fact, your spell
points depend only on the amount of jewels you invest in magical R\&D.

Jewels are also important in that they constitute your nation's
reserve that backs up its currency and bonds.  When you issue bonds,
these have to be backed up by jewels, so it is a good idea to save
some jewels, and not spend them all. (note: bonds are not yet
implemented)

(We must find another use for jewels, so nations have more choices to
make in spending jewels.)

\section{Transportation and trade}
You can trade with other nations or just transport goods/armies/people
for your own benefit, using {\em ships} and {\em caravans}.  Caravans
travel on land, whereas ships travel in the water. (For water based
nations, the inverse is true.)

Ships and caravans are drafted as if they were armies, and appear in
your [l]ist of available armies in the [a]rmy menu.  The construction
and maintainance costs of ships and caravans are tabulated with those
of other army types.  Note that some spirits also behave as ships and
caravans, in that they have the {\em cargo} flag.  Examples of this are
the {\em flying carpet}, the {\em ghost ship} and the {\em living
ship}.

A single cargo hold can only transport a certain amount of goods.  The
unit of weight is {\em the weight of a single person}, and a caravan
can transport 250 person weights.  A bar of metal weighs 0.1, money
weighs 0.01 for a sheckel, food 0.05, a jewel basket 0.01.  If you
load soldiers, their weight is equal to the weight of the number of
people plus the weight of the metal used in drafting the army.  Any
caravan can transport only a single army and a single land title.  The
land title does not have significant weight.

To load a caravan or navy, you select it (with [a]rmy commands) and
then use the [t]ransportation command to [l]oad goods, which can be
[s]hekels (money units), [m]etal, [f]ood, [p]eople, [a]rmy or
[t]itle.  To unload it you move the caravan or navy to its destination
and do the same with [u]nload instead of [l]oad.

You can only load certain goods onto a caravan in certain places.
Anything can be loaded in a city (or your capital).  Metal can also be
loaded in metal mines, jewels in jewel mines and food in farms.  The
title to a sector {\em must} be loaded in that sector itself.  People
can be loaded from any of your own populated sectors.  Armies can be
loaded anywhere in your land, but out of your territory they can only
be loaded if they have the front-line flag.

To unload in a foreign land you must be in a city or town.  The only
exception is that you can unload armies with the front-line flag
anywhere.  To {\em trade} an army in a foreign city or town, you
should put the army in {\em traded} status, and then unload it in the
city or town.

Alternatively, since armies move on their own, you can put them on
{\em traded} status and just walk them up to the city or town.  When
you stop that army on the foreign city or town, you will be asked if
you really want to trade it.  You can also change the army to {\em
traded} status once it is already at the city or town.

If you have unloaded goods or armies into a foreign nation's city or
town, the goods or armies will become theirs.  During the update, each
nation will receive mail which indicates what trades occurred.

Transporting goods within your country is not really useful.  However,
transporting people is an effective way of getting them to the better
mines and farms.  Transporting armies can help mobilize your forces
more quickly, since you can then unload them and they can still march.

Some armies, such as Sailors and Marines, have the {\em front line}
flag (see the army types table).  This means that they can be unloaded
from caravans and ships anywhere: in your land, in un-owned land, and
in foreign land.  Thus ships and caravans can be used for
transportation of fighting troops, not just for trade and migration.

If a transport (caravan or ship) is on a sector where there is a
battle, it will be destroyed if more than 85% of its accompanying
force is killed.  Otherwise it will be left intact.  If an army is on
a transport and a battle occurs, the army's bonus is modified by the
bonus of the transport.  Thus, if 100 soldiers with bonus 30 are on a
sailboat which has bonus -50, and are in a battle, their bonus will be
-20.

\section{Communications}
Nations in dominion communicate through {\em mail} and
{\em news}.  Mail allows you to send personal messages to leaders
of other nations.  News is for general announcements, and is read by
all nations.

There can be several newsgroups.  One is always reserved for messages
from the computer, containing general information on what has happened
over the update.  This newsgroup is usually called ``News''.  Other
newsgroups are set up by the Game Master, and any nation leader can
post to them.  At Stony Brook we usually have a newsgroup called
``public'' which receives many very creative postings from
participants.

You should read your mail and news whenever you play your turn, to be
in touch with your neighbours and the rest of the world.  You also get
mail from the update program after each update, telling you what has
changed in your nation over the update.  Alternatively, you can tell
dominion to forward your mail, news, or both, to an electronic mail
address using the [O]ptions menu.

\section{World Geography}
The world is shaped like a torus (i.e. the surface of a doughnut).
This is the best approximation of a sphere which can be displayed
easily on a flat terminal.

Thus, the world wraps in both the north-south and east-west
directions. So, if you are playing in a 100x100 world, and your nation
grows to be 100 sectors wide, you can travel around the world.

\subsection{Terrain}

There are various types of terrain.  They affect how fast a race can
move on a sector, and are related to the food production of that
sector.  We define {\em terrain} to mean what covers the surface of
that sector.  You find a sector's terrain described in the sector
window, in the sector [Z]oom window, and in the [t]errain [d]isplay.
Table \ref{tab-terrains} lists the possible terrains in Dominion,
together with the marks that appear in the [t]errain [d]isplay.

\begin{same}
\begin{table}[hbpt]
\caption{Terrains}
\label{tab-terrains}
\begin{tabular}{ || l | l | l | l | l | l ||}
\hline
Description & mark & Description & mark & Description & mark \\
\hline
Ocean       &  O   & River       &  R   & Grassy      &  g   \\
Bay         &  B   & Ice         &  #   & Brush       &  B   \\
Reef        &  R   & Barren      &  b   & Forested    &  f   \\
Lake        &  L   & Swamped     &  s   & Jungle      &  j   \\
\hline
\end{tabular}
\end{table}
\end{same}

\subsection{Altitude}

A sector's altitude greatly affects how fast races move on that
sector.

The altitudes in Dominion range from Ocean Trenches to Mountain Peaks.
You find a sector's altitude described in the sector window, the
sector [Z]oom window, and the [A]ltitude [d]isplay.

You also see the altitude marks for un-owned lands on the ordinary
designation map display (the default display when you start playing).

Table \ref{tab-altitudes} lists the altitudes together with the marks
that appear on the map and the numbers that give the ordering.  Notice
that altitude 0 is sea level, and is considered land.

\begin{same}
\begin{table}[hbpt]
\caption{Altitudes}
\label{tab-altitudes}
\begin{tabular}{ || l | l | r | l | l | r ||}
\hline
Description   & mark & level & Description   & mark & level \\
\hline
Trench        &  v   &  -5   & Lowlands      &  .   &   1 \\
Ocean Plains  &  -   &  -4   & Plains        &  -   &   2 \\
Sea Mountain  &  ^   &  -3   & Hills         &  %   &   3 \\
Cont. Shelf   &  +   &  -2   & Plateau       &  =   &   4 \\
Shallows      &  #   &  -1   & Mountains     &  ^   &   5 \\
Sea Level     &  ~   &   0   & Mountain Peak &  +   &   6 \\
\hline
\end{tabular}
\end{table}
\end{same}

\subsection{Climate}

Climates in Dominion range from Desert to Polar.  They have a
significant effect on each race's speed of movement, and they also
affect the sector's food productivity.  Climates can be seen with the
[w]eather [d]isplay, or in the sector [Z]oom window.

Table \ref{tab-climates} shows all the climates together with the
marks that appear in the [w]eather [d]isplay.

\begin{same}
\begin{table}[hbpt]
\caption{Climates}
\label{tab-climates}
\begin{tabular}{ || l | l | l | l ||}
\hline
Description            & mark & Description            & mark \\
\hline
Desert                 &  D   & Mid-latitude Marine    &  m \\
Semiarid               &  d   & Humid Continental Warm &  w \\
Humid Subtropical      &  h   & Humid Continental Cool &  c \\
Tropical Wet and Dry   &  t   & Subarctic              &  a \\
Rainy Tropical         &  T   & Polar                  &  A \\
Dry Summer Subtropical &  s   &                        &    \\
\hline
\end{tabular}
\end{table}
\end{same}

\chapter{Diplomacy and war}

Diplomatic relations with your neighbours are extremely important.
Your nation could be destroyed if you do not properly set your
diplomacy: you might make several enemies who could then form a treaty
to fight you.  This happens quite often.

Your nation starts out with 10 armies of 100 Cavemen each in your
capital; more can be drafted using the [a]rmy menu.  Soldiers are used
to occupy unowned land, to defend your own territory, and also to
conduct war against enemy nations.

To occupy an unowned sector, you must have an army of at least 100
soldiers there, set on occupy mode.  To occupy a sector owned by
another nation, you have to declare {\em war} or {\em jihad} with
them.  Sectors can also be occupied by spirits with 100 units or more.
If you occupy an enemy's sector that has people on it, they will be
hostile toward you for one thon.  This means that you will not be able
to draft from the sector or redesignate it.

\section{Diplomatic status}
\label{sec-diplomacy}
In the [r]eports menu, you can access your [d]iplomacy report.  This
report shows your status toward other countries, and their status
toward you.  You start out {\em UNMET} with all nations.  Then, as
your armies come close to their sectors, or vice-versa, the two
nations will meet and be put in neutral status.

You can change your status toward other countries as you meet them.
Many statuses are possible, but the most important ones are:
\begin{itemize}
\item
\strong{Allied}: 
this gives permission to the other nation to pass through your land at
a lower move cost.  Also, you can put your armies in GARRISON in
allied land, and they will get 1/2 of the GARRISON bonus.
\item
\strong{Treaty}:
this goes beyond ALLIED: if your armies or those of the other nation
are involved in a battle, and the other has armies on the same sector,
the two will fight together.  Also, if you put your armies in GARRISON
in treaty land, they will get the full GARRISON bonus.
\item
\strong{War}:
if any army of yours is on the same sector as an enemy army, and one
of the two is on ATTACK or OCCUPY mode, there will be a battle.
\item
\strong{Jihad}:
for now, this is the same as WAR.  In the future, JIHAD should involve
some expense, and give a better fighting bonus due to fanatism in
combat.
\end{itemize}

You can change your diplomatic status towards any nation you have met.
They will see the change immediately.  You can only change it by two
degrees each thon, so that you cannot be ALLIED, march into someone's
land, and then declare WAR and occupy all their sectors.

\section{Armies}
The {\em army types} table lists the various types of armies, their
costs, bonuses and move rates.

Each army or spirit has a set of {\em army flags} that affect its
behaviour.  Some are innate abilities of that army, others can be set
with magic spells.  The table lists each army's innate flags.  The
flag abbreviations are:
\begin{itemize}
\item
\strong{F - Flying}: Any army with this flag can fly.  Move costs of sectors
don't depend on altitude, and thus are much lower.  These armies also
ignore PATROLs and INTERCEPTs unless the patrolling army has missiles.
\item
\strong{H - Hidden}: An army with this flag is magically cloaked.  Other
nations will not be able to see the army on their maps.
\item
\strong{V - Vampire}: An army with this flag will possess some of the
dead on the battlefield who will join ranks with the army.
n\item
\strong{T - Transport}: This army is being transported as cargo.
\item
\strong{\^\ - Missiles}: This army has weapons that are thrown, such as
arrows. Archers are a type of army with missles.  These units, when
in PATROL or INTERCEPT status, will slow down flying units too
(normally flying units are not slowed down by PATROLs or INTERCEPTs,
though they {\em do} get intercepted once they land).
\item
\strong{W - Water}: An army with this flag must stay in water, or in the land
sectors along a coast; even if it belongs to a land nation.  Compare
with the L flag below.
\item
\strong{L - Land}: This is the exact inverse of the W flag.  An army with
this flag must stay on land, or in water sectors along the coast; even
if it belongs to a water nation.  Notice that if you have both the
L and W flags, you can move that army onto {\em any} sector on the
map.
\item
\strong{I - Inverse altitude}: This flag says that the army will have either
the L or the W flag, depending on the race of the army's owner.  If
the army belongs to a {\em land} race, then it will have a W flag,
and vice-versa.  This flag is used mostly for things like ships,
which travel in the ``inverse'' altitude medium.
\item
\strong{f - Front-line}: Army can be loaded/unloaded on/from a ship or
caravan on land that does not belong to you.
\item
\strong{k - Kamikaze}: Armies with this flag will fight with very high bonus,
but will all die in any battle.
\item
\strong{m - Machine}: These armies are really machines, such as siege engines,
war carts and catapults.  They help ordinary armies in combat by
destroying fortifications and adding bonus to the armies.
\item
\strong{d - Disguised}: You can disguise this army so that it will appear to be
of another type to other players.  For example, to disguise a
poltergeist as an areal serpent you would change the army's
{\em name} so that it ends with ``/areal_serpent''.
\item
\strong{w - Wizard}: This flag means that the army can summon spirits
and cast spells, like a mage.
\item
\strong{s - Sorcerer}: This flag means that the army can excercise powers
of sorcery (not implemented yet).
\item
\strong{c - Cargo}: This army can transport cargo, like ships and caravans.
\item
\strong{U - Underground}: This army can burrow underground, and thus is not
slowed down by patrols and intercepts.
\item
\strong{u - Uncontrolled}: This means you have lost control of the army or
spirit, usually by not having enough spell points available to support
a spirit.  Uncontrolled armies have no movepoints, have their status's
set to neutral, and will not prevent an opposing force from taking a
sector.  As well uncontrolled armies will be dibanded in not brought
under control within one thon.
\item
\strong{R - Race specific}: These armies are available only to certain races.
Race specific armies have a fixed bonus related to their original race
and are unaffected by improving technology, and sometimes they have
special flags (for example, Hobbits are hidden).  They are available
from the start of the game, and are much more powerful than Cavemen,
so they present a strong advantage.  When traded, these armies keep
their special characteristics and their bonus; they will not acquire
the strength of the nation to which they are traded.
\item
\strong{* - Magic order specific}: These armies are available only to certain
magic orders.
\end{itemize}

\section{Army types}

Table \ref{tab-army-types} is the {\em Army types} table, which
describes in detail all army types available for drafting.  Some of
the army types deserve a special description here:

\begin{same}
\begin{table}[hbpt]
\caption{Army types table}
\label{tab-army-types}
\begin{tabular}{ || l | l | r | r | r | r | r | r | r | r | l || }
\hline
Type      &ch&move&bonus&draft&draft&draft&maint&maint&maint&flags \\
          &  &rate&     &money&metal&jewel&money&metal&jewel&      \\
\hline
Cavemen      &*& 0.5& -60 &  60 &   0 &   0 &  10 &   0 &   0 &\\
Caravan      &C& 1.0&   0 &1000 &1000 &   0 & 200 &  40 &   0 &c\\
Mage         &!& 2.0&   0 &   0 &   0 &5000 &   0 &   0 &1000 &w\\
Spearmen     &/& 0.8& -10 &  80 &  20 &   0 &  10 &   0 &   0 &\\
Infantry     &i& 1.0&   0 & 100 &  30 &   0 &  10 &   0 &   0 &\\
Archers      &)& 1.0&   5 & 100 &  40 &   0 &  10 &   0 &   0 &^\\
Canoes       &u& 2.0& -80 &1000 &2000 &   0 & 200 & 100 &   0 &cI\\
Swimmers     &o& 1.0& -60 & 150 &  50 &   0 &  15 &   5 &   0 &Wf\\
Walkers      &#& 1.0& -60 & 150 &  50 &   0 &  15 &   5 &   0 &fL\\
Phalanx      &p& 1.0&  20 & 150 &  50 &   0 &  12 &   0 &   0 &\\
Sailors      &~& 0.0& -10 & 150 &  50 &   0 &  15 &   0 &   0 &f\\
Chariots     &0& 1.5&  10 & 180 &  60 &   0 &  14 &   5 &   0 &\\
Legion       &l& 1.0&  20 & 120 &  30 &   0 &  10 &   0 &   0 &\\
Cavalry      &c& 2.0&  20 & 200 &  80 &   0 &  20 &   0 &   0 &\\
Elite        &e& 1.3&  10 &   0 & 100 &   0 &   0 &  25 &   0 &\\
Sailboats    &\}& 2.5& -50 &1500 &3000 &   0 & 300 & 150 &   0 &cI\\
Marines      &m& 1.0&  60 & 300 & 200 &   0 &  25 &   0 &   0 &f\\
Wagons       &W& 1.5&   0 &1500 &1500 &   0 & 300 &  60 &   0 &c\\
War_carts    &w& 1.0&  10 & 500 & 200 &   0 & 200 &  20 &   0 &m\\
Galleys      &g& 3.5& -20 &2000 &4000 &   0 & 400 & 200 &   0 &cI\\
Berzerkers   &b& 1.0&  50 &  30 &   0 &   0 &   5 &   0 &   0 &k\\
Merc         &M& 1.2&  60 & 200 &   0 &   0 & 200 &   0 &   0 &\\
Catapults    &@& 0.5&  20 & 800 & 400 &   0 & 300 &  30 &   0 &m\\
Quadriremes  &q& 4.5&   0 &3000 &5000 &   0 & 500 & 250 &   0 &cI\\
Scuba_divers &S& 1.2& -30 & 300 & 200 &   0 &  25 &  25 &   0 &fI\\
Kamikaze     &k& 1.5& 200 &  50 &  25 &   0 &  50 &   0 &   0 &k\\
Ninja        &N& 3.0&  75 &2000 &   0 &2000 & 500 &   0 & 500 &H\\
Crossbowmen  &]& 1.0&   5 &  50 &  60 &   0 &   5 &   5 &   0 &^\\
Orcs         &O& 1.0&   0 &  60 &  30 &   0 &  10 &   0 &   0 &R\\
Hobbits      &H& 1.0&   0 & 100 &  30 &   0 &  10 &   0 &   0 &HR\\
Harpies      &y& 0.7& -10 & 100 &  30 &   0 &  50 &   0 &   0 &FR\\
Ogres        &G& 1.0&  50 & 100 &  30 &   0 &  10 &   0 &   0 &R\\
Hunters      &h& 0.8& -10 &  50 &   0 &   0 &  10 &   0 &   0 &*\\
\hline
\end{tabular}
\end{table}
\end{same}

\begin{itemize}
\item
\strong{Legion}: A sophisticated form of Infantry with much higher bonus.
They become available with higher technology and cost barely more than
infantry to draft; metal and maintainance are the same.  Once you have
Legion available, there is no reason to continue drafting Infantry.
\item
\strong{Phalanx}: Not yet very well inserted into Dominion.  They should in
principle have greater strength in numbers, but this is not yet
implemented.
\item
\strong{Elite}: Cost a lot of metal and no money to draft and maintain;
they are best used if you have too much metal and are bankrupt.
\item
\strong{Ninja}: Good for spying, but cost too much in jewels to make
them effective as large armies.
\end{itemize}

\section{Army maintenance costs}
Each army has a maintenance cost in money, metal, and/or jeweles,
which is for the salaries of the soldiers and upkeep of weapons and
other materials needed by the army.  The maintenance cost for each
army is given in the {\em army types} table.  There is also a fixed
per-army overhead of 2000 sk.

\section{Movement points}
Your armies can move only a certain distance before they must stop and
rest.  How much they can move is expressed in the army's {\em move
points}, and the {\em move cost} of the sectors they cross.

For example, if an army has 13 move points, and it goes through
sectors with move costs of 4, 3, and 4, it will have two move points
left.  It will only be able to move onto a sector with move cost of
two or less.

\section{Army statuses}
\begin{itemize}
\item
\strong{[i]ntercept}: This army will intercept a nearby enemy army.
Intercept raises the move cost of adjacent sectors, and then (over the
update) the army will move to any adjacent sector where an enemy army
might be located.  This is quite powerful, since you don't have to
defend each square, but you should be careful because intercepts can
be decoyed.  An army must have at least 50 soldiers to intercept.

Intercept will also move the army to intercept a flying enemy army,
once it has landed.  But the increased move cost will not apply to flying
enemy armies {\em unless} the intercepting army wields missiles (for
example, archers).  See the description of army flags.

\item
\strong{[p]atrol}: This army will patrol the area, raising the move cost
of adjacent sectors for enemy armies.  If the enemy army is flying,
the move cost will only be raised if the patrolling army has missiles
(for example, archers).  An army must have at least 50 soldiers to
patrol.

\item
\strong{[g]arrison}: This army garrisons the sector, getting an extra
20 bonus on your own land.  If the sector belongs to a nation at {\em
treaty} with you, then you also get bonus 20.  If it is {\em allied}
land, you get a bonus of 10.

\item
\strong{[n]eutral}: This army is neutral and will not involve itself in
any conflicts that occur.  You cannot choose this status, it is 
assigned automatically to any army or spirit that becomes uncontrolled.
\end{itemize}

\comment \section{Tactics}

\chapter{Detailed description of commands}

\section{Options menu}
You can set some options that affect the way the game appears to you.
The [O]ptions menu allows you to set the following options:
\begin{itemize}
\item
\strong{e[x]pert mode}: This toggles expert mode on and off.  Expert mode
allows an experienced player to do things much more quickly by showing
most menus on the status line, instead of drawing windows.
\item
\strong{[f]orwarding mail}: This allows a you to get mail forwarded
to your e-mail account instead of reading mail inside dominion.
\item
\strong{[n]ews forwarding}: This allows a you to have all of the
news posted to the game to be forwarded to the mail address you specify.
You will still be able to read news inside dominion.
\item
\strong{[c]ivilian movement}: This toggles between the three available
types of migration that your government allows: {\em Free},
{\em Restricted} and {\em None}.  These are described in section
\ref{sec-migration} on migration.
\item
\strong{[m]ail program}: This allows you to choose a mail program with
which to read your dominion mail.  Examples are ``elm'', ``mush'',
``Mail'', or ``mailx''.  If you type nothing, you will get the built-in
bare-bones mail program.
\item
\strong{[e]ditor program}: This allows you to choose your default editor
and to overide any environment variables you may have set.  If the
gamemaster has a restricted list of usable editors then it will tell
you which editors you may choose from.
\end{itemize}

\section{Display menu}
The [d]isplay menu allows you to change what information is given on
the sector map.  The default view of the screen shows the designation
of your own sectors, altitude markings for unowned sectors, and nation
marks for other nations' sectors.  If you are a land nation, all water
sectors are marked as a ``~'', and if you are a water nation, land
sectors are marked by a ``.''.  Land nations may view underwater
sectors, and vice versa, with the [W]ater toggle option in the
[d]isplay menu.  There are many other ways of looking at the map, and
of highlighting sectors on it.  These are described briefly here.
\begin{itemize}
\item
\strong{Map Style}
\begin{itemize}
\item
\strong{[c]ompact}: This will remove the spaces between sectors in the
display, allowing you to see twice as much on the map.
\item
\strong{[r]egular}: This will display the map with a space between sectors.
\end{itemize}
\item
\strong{Display Options}: Most of these simply display characters or numbers
based on which you choose (for example, {\em [n]ation mark} will
display letters indicating the nation which owns the sector).  Numbers
greater than 9 are displayed as a plus sign (for example, metal 12). A
few options require further explanation:
\begin{itemize}
\item
\strong{[p]opulation}: For sectors containing less than 950 people, this
displays a number indicating the rounded number of people divided by 100.
Thus a sector with 475 people will be displayed with a ``5''.  For sectors
with more than 950 people, an ``I'' is displayed.
\item
\strong{[T]errain move cost}: This shows the basic move cost of each sector,
with regard to your race.
\item
\strong{army [M]ove cost}: This shows the move cost for each sector, after
adjustments are made for the army you have selected.  For example, if your
army flies, the move costs will be much lower.
\end{itemize}
\item
\strong{Highlight Options}: These select which sectors will be highlighted
in the map (they will be displayed in reverse-video or in a different
color depending on your terminal).
\item
\strong{Other Things}: [C]enter screen at cursor will do just that, and
the [W]ater toggle was explained at the beginning of this section.
\end{itemize}

\section{Movement}
The movement commands are shown in the diagram at the beginning of
this manual.  They are quite straightforward, and only behave
differently when you reach the edge of the screen.  In this case, the
screen is shifted over.  The screen can be forcibly centered around
the cursor using the [d]isplay [C]enter screen command.

You can also jump directly to a sector with the [p] command.  You will
be prompted for the coordinates to jump to.  You can jump back to your
capital with the [P] command.

\section{Reports}
Here is a detailed description of each report in dominion. To access
them type [r]eports and then the letter of the report you want. You
can switch from one report to another by hitting the key that
corresponds to the report you want.  The reports you can switch to are
listed at the bottom of the report you are viewing.

\subsection{Information report}
To access this report you type [r]eports followed by [i]nfo.

This report gives complete information about your nation.  Below is a
description of most parameters printed.  Simple parameters such as
{\em Nation} are left out.

\begin{itemize}
\item
\strong{Treasury, Jewels, Metal, Food}: Your current wealth.
\item
\strong{Birthrate}: The percentage by which your population increases
each thon.
\item
\strong{Mortality}: The percentage of people who die in your nation every thon.
\item
\strong{Intelligence}: This affects your acquisition of technological skill.
It can be increased by building universities.
\item
\strong{Magic Aptitude}: This affects how quickly you learn the spells for
your nation's magical order, and how many spell points you gain.  It
is based on your race.
\item
\strong{Magic Skill}: What level of magic your nation is capable of.
Investment in magic R\&D increases this.
\item
\strong{Speed}: Your armies' move rate is proportional to this.
\item
\strong{Spell points}: Tells you how many you have and how many are spent for 
maintenance of spirits.
\item
\strong{Technology skill}: What level of technology your nation has reached.
Investment in technological R\&D increases this.
\item
\strong{Farming skill}:  The higher this is the more food you produce
per farmer.  This is based on your race and can be increased by new
technologies.
\item
\strong{Mining skill}: This affects how productive your metal and jewel
mines are.  This is based on your race, and can be increased by new
technologies.
\item
\strong{Spy skill}: This expresses how good your intelligence service is.
\item
\strong{Secrecy}: This indicates how resistive you are to espionage by
other nations.
\item
\strong{Combat bonus}: This percentage is added to the number
of soldiers in your armies to get the effective force your army fights
with.
\item
\strong{Move points}: This is the basic distance each army can move
each thon. 
\end{itemize}

From here you can change your [P]assword or [l]eader name.  You can
[t]oggle your nation to or from CN status, which will cause the
computer to play your turns for you.  This will also let you choose if
your nation should receive update mail while it is being run by the
computer. Changing your nation to an CN is a risky move: the computer
will play a good game, but it will not honor your long-term plans.

\subsection{Budget report}
\label{sec-budget-report}
To access this report you type [r]eports followed by [b]udget.

The budget report gives you detailed breakdown of how you are spending your 
money and your natural resources.  

Within the budget report you can adjust what percent of your money
and/or natural resources you are spending on research in technology,
the study of magic, and reconnaisance. You also can adjust your tax
rate.

The screen shows how much money you are spending on military
maintenance and other costs inccurred. The only way military
maintenance can be lowered is to disband armies. Other costs include,
but are not limited to the cost of drafting an army and the cost of
redesignating sectors.

The metal and jewels breakdown lists the amounts spent on research and
development, from both current revenue and storage, and the amounts
spent by other activities.  Other metal expenses include the
construction of citites and the drafting of armies.  Other jewels are
used as a maintenance fee for mages.

The commands within this screen are:

\begin{itemize}
\item
\strong{[t]ax rate}: Lets you adjust your tax rate.
\item
\strong{[T]ech R\&D}: You choose how much of your metal or money
revenue is invested in technology research.
\begin{itemize}
\item
\strong{[m]etal} sets the amount of metal you wish to invest.
\item
\strong{[M]oney} sets the amount of money you wish to invest.
\end{itemize}
\item
\strong{[M]agic R\&D} - You choose how much of your jewels or money
revenue is invested in magic research.
\begin{itemize}
\item
\strong{[j]ewels} changes the amount of jewels you invest.
\item
\strong{[M]oney} adjusts the amount of money you invest.
\end{itemize}
\item
\strong{[S]py} invests money in your {\em spy department}
\item
\strong{[s]torage} allows you to spend money/metal/jewels from your storage.
The way you can spend it is similar to how you spend your revenue, but
it is a once-only expense, and is cleared after each update.
\end{itemize}

\strong{A note of caution}: Watch how much money you are spending
carefully; you can run your nation at a deficit, but if you have no
money, you cannot draft, construct or redesignate.  If someone attacks
you and you need armies quickly, you will be in trouble.

\subsection{Production report}
To access this report you type [r]eports followed by [p]roduction.

The production report tells you how many people are employed and
unemployed in each area of your economy.  It also shows you the
average productivity of each employee.  Your total economy is listed
as {\em General}, and below it is broken down into {\em Metal miners,
Jewel Miners, Farmers}, and {\em Services}.  Services include all
other types of employment, such as workers in cities, forts, etc.

\subsection{Nations report}
To access this report you type [r]eports followed by [n]ations.

This report displays minimal information on each nation in the world.
The report lists the nation id, nation name, nation mark, leader name,
and race of each nation.  If the nation is currently being played by
the computer, ``cn'' is displayed along with the information.  If the
nation has been destroyed, that will also be indicated.  You are also
told the size of the world and how many nations are in it.

Within this report is the [s]py option. If you select this, you will
be prompted for a nation id, and then you will be presented with a
screen that allows you to bribe officials in that nation for
information.

    You pay a certain amount of jewels in bribes, of which a
percentage equal to the tax rate of the nation being spied
upon ends up in that nations coffers. Certain types of information
are easier to obtain than others.  In order of increassing difficulty:
[C]apital location, [t]echnology level, [e]conomy, ma[g]ic skill,
[p]opulation, and [m]ilitary.  The accuracy of the information recieved
depends on the type of information sought, comparison of your spy
skill to their secrecy skill, and upon the amount of jewels
spent as a percentage of the nation being spied upon's jewel
supply.  And with each attempt at spying made there is a small
chance that the spy will be noticed by the other nation.

    You can also steal technology from another nation, and 
while the factors involved are still the same, it is much more
difficult, the spy is much more likely to be caught, and if 
you don't spend enough in bribes then it is almost certain that
you'll get no results.  Lastly, the amount of tech stolen depends
also on the difference in your tech levels.  It is much easier
to steal technology from nations vastly superior to your own than
from those close to your level.

\subsection{Diplomacy report}
To access this report you type [r]eports followed by [d]iplomacy.

Diplomacy with other nations is extremely important.  You should set a
status with each nation you have met.  This is done in the
{\em diplomacy} report.

In most cases, all the nations in a dominion game will not fit on the
screen at once.  By pressing [>] or [.] you can move to the next page
of nations.  Pressing [<] or [,] will go back a page.  Once you have
found the nation you wish to change your status to, press [c]hange
status and enter the nation number.  You will be presented with a list
of statuses to choose from.

Be sure to read section \ref{sec-diplomacy} on diplomatic statuses
for more information on what the statuses mean, and how to change
statuses.

\section{Construction}
You can construct on a sector with the [C]onstruct command.  Your
construction will cost money and/or metal, and can make that sector
more valuable.  The construct menu options are:
\begin{itemize}
\item
\strong{[r]oads}: Roads decrease the move cost for you (and anyone else)
in the sector.  The cost for building roads doubles for the next
level of road construction.  For each level of roads construction, the
move cost goes down by 1 (but it never goes below 1).
\item
\strong{[f]ortification}: Fortifies the current sector: adds 10 to the
fortification level, which gives your armies or armies of treatied
nations that much bonus when defending that sector, and allied nations
one-half that bonus.  Many fortifications can be built on a sector
and their effect is cumulative.  The cost doubles for each level of
fortification.
\item
\strong{[b]ubbles}: These are air/water-tight bubbles.  They are necessary
to colonize underwater (if you are a land race) and land (if you
are an underwater nation).  Once you have a bubble, you can move
troops and civillians to that sector.  Only one bubble can be constructed
on a sector (that is all you need).
\item
\strong{ref[i]nery}: A refinery is constructed in a metal mine, and
will increase the productivity of the mine by 12% by refining the metals
produced there.  Only one refinery can be constructed on a sector.
\item
\strong{[d]estroy}: This will bring up a sub-menu that allows you to
remove any of the above.  Removal of construction will cost you 
money, but you will also be able to retrieve a small percentage of
the metal building costs.
\end{itemize}

What you can construct on a sector, and how much of it you can
construct, is based on your technology level.  As your technology
improves, you will gain new skills which let you construct more.

\section{Wizardry}
\label{sec-wizardry}
Type [W]izardry to enter the wizardry menu.

The wizardry command has five options.  These are:
\begin{itemize}
\item
\strong{[l]ist spells and spirits} - This will show you which spirits are
available to summon and how many spell points they cost, and which
spells you can cast, how much they cost, and how long they last (in
thons).
\item
\strong{[c]ast a spell} - This lists the available spells, and asks you
to enter the name of the spell you wish to cast.  If the spell affects
a sector (such as {\em caltitude}) it will affect the sector your mage
is on.  If it affects an army (such as {\em hide_army}), you will be
asked to enter the army id.  The army must be on the current sector.
\item
\strong{[s]ummon a spirit} - This lists the available spirits, and asks you to
enter the name of the desired spirit.  The spirit will appear on the
current sector.  Spirits may only be summoned in your own land.
\item
\strong{[h]anging spells} - This lists all spells you have cast which have not
yet expired, and indicates how much longer the spell will last. For
example, the {\em hide_sector} spell lasts for eight thons, so would
appear in the list until that time is up.  In this list you may [z]oom
on a specific spell, which will give more detail, such as which sector
the spell affects.  You may also [d]elete a spell which will terminate
it early.
\item
\strong{[i]nitiate a mage} - Mages are needed to cast spells or summon spirits.
They cost jewels to initiate and maintain, and move twice as fast as
regular armies.
\end{itemize}

\chapter{The update}

It can be useful to understand exactly what happens during the update.
Here are the steps and the order in which the computer performs them.

\begin{enumerate}

\item The current world is loaded.
\item Each nation is loaded.
\item Hanging spells are loaded
\item Trading via cargos is handled.
\item CN moves are made.
\item Technology, Magic, and Spy skills are updated.
\item Revenue of money, metal, jewels and food are calculated.
\item Civilian migration occurs.
\item Battles are resolved.
\item Sector capture is handled.
\item Diplomacy is updated (nations become ``met'')
\item Armies are reset.  Mages are disbanded if maintaince jewels are missing.
\item Mail is sent to each nation and statistics are posted to news.
\item The new world file is saved.

\end{enumerate}

\section{Migration}
\label{sec-migration}

Civilian migration happens automatically.  The people move according to
the laws of the country.  As a leader you can set those laws with the
[o]ptions screen:  you can set migration to be {\em Free} (the default),
{\em Restricted} or {\em None}.

\begin{itemize}
\item
\strong{Free}: People look for the sectors that have the most
available jobs, because that's where they will get the best pay/best
job.  Some attraction is also excersized by how pleasant a sector's
living conditions are, but in general a constant ratio of employment
is preserved locally.
\item
\strong{Restricted}: The govevernment (you) has gotten sick of all these
civilians wandering around so much.  So civilians will not be allowed
to leave their present place of residence unless they can prove they
are unemployed.  If so then they will move to the sectors surrounding
them, partially by number of jobs available, partially by the
desirability of the sectors available.  However all sectors in range
will be put to full employement if possible.  If there is nowhere to
go they will wander aimlessly in search of a place to find a job,
jumping with the latest rumor of employment.
\item
\strong{None}: The government has decreed that there will be no
civillian movement.  Where they are is where they stay, and where they
are born is where they will die.  Unless the army comes to get them in
caravans or ships, they will stay put.
\end{itemize}

\chapter{Authors}
Version 1.02 of Dominion was the first with this name.  It was
previously called {\em Stony Brook World} (SBW), until too many people
suggested a catchier name.

Here is a list of the people who actually wrote code for sbw/dominion
that is in the current release.  The order is that in which they wrote
their first piece of code.

Mark Galassi (rosalia@dirac.physics.sunysb.edu) User interface (in
curses), basic data structures, world generator, economy, technology,
magic, basic army work, manual and formatting of manual with
LaTeXinfo, miscellaneous.  Currently maintains dominion.

Michael Fischer (greendog@insti.physics.sunysb.edu) Update program,
battle code, developed exec file format, dom_print, manual, lots of
other things.

Doug Novellano (doug@insti.physics.sunysb.edu) Mail and News systems.

Keith Messing (keith@max.physics.sunysb.edu???) Initial diplomacy system.

Alan Saporta (gandalf@insti.physics.sunysb.edu) Work on some exec
routines, many suggestions of directions for the game.

Joanne Rosenshein (raven@max.physics.sunysb.edu) Initial draft of the
manual, many suggestions of directions for the game.

Stephen Bae (sbae@max.physics.sunysb.edu???) Basic world memory
allocation.

Chris Coligado (noel@max.physics.sunysb.edu???) Army and battle code.

C. Titus Brown (brown@reed.edu) Adding nations and improvements on the
reports; revised army menu and transportation menu.  Lots of
miscellaneous things.

Charles Ofria (charles@krl.caltech.edu) Designed the ``cns'' file,
most magic orders and many races and techno powers; coded some spells.

There are also some major contributions from people not in Stony
Brook:

Stephen Underwood (hiranu@netcom.com) Fractal terrain generator and
contributions in very many areas.

Paolo Montrasio (montra@ghost.unimi.it) Several suggestions.

Kevin Hart (hart@cs.utk.edu) CN system.

Many others have made very important creative suggestions to the game,
though they were not involved in the actual coding.  Here are some
names that come to mind.  Please send us mail if we have forgotten
any.  Tony Matranga, Tim Poplaski, Chris Adami, and everyone else who
participated in the FALL SBW and SPRING DOMINION games at Stony Brook
in the fall 1990 and spring 1991 semesters.

\appendix

\chapter{Spirit tables}

\label{app-spirits}

The {\em Spirit types} tables below list the spirit types available
in dominion.  Summoning costs (in spell points) are not listed and may
be different if a spirit is available to more than one magic order.
The flags for each spirit are described in the section on
{\em Armies} above.

\begin{same}
\begin{table}[hbpt]
\caption{Spirit types table (Aule)}
\begin{tabular}{ || l | l | l | l | l | l || }
\hline
Name            & Size &Move&Bonus&Spell&Flags\\
                &      &rate&     &Pts  &\\
\hline
gargoyle        &   30 & 1.0&   0 &   9 &F\\
mole            &   70 & 1.0&   0 &  15 &UL\\
dust_devil      &  150 & 1.0&   0 &  27 &FL\\
umber_hulk      &  250 & 1.7&   0 &  40 &UL\\
stone_giant     &  500 & 0.7&   0 &  51 &\\
earth_elemental & 2000 & 2.0&   0 &  173 &U\\
mountain        & 3000 & 0.3&  60 &  253 &\\
\hline
\end{tabular}
\end{table}
\end{same}
\begin{same}
\begin{table}[hbpt]
\caption{Spirit types table (Avian)}
\begin{tabular}{ || l | l | l | l | l | l || }
\hline
Name            & Size &Move&Bonus&Spell&Flags\\
                &      &rate&     &Pts  &\\
\hline
flying_carpet   &    1 & 1.5&   0 &   5 &Fc\\
roc             &   30 & 1.5&   0 &   8 &F\\
eagle           &   70 & 1.5&   0 &  14 &F\\
cloud_giant     &  150 & 1.0&   0 &  21 &F\\
wyvern          &  250 & 1.0&   0 &  31 &F\\
areal_serpent   &  500 & 1.8&   0 &  59 &F\\
air_elemental   & 2000 & 2.0&   0 &  162 &F\\
tempest         & 3000 & 2.0&   0 &  217 &F\\
\hline
\end{tabular}
\end{table}
\end{same}
\begin{same}
\begin{table}[hbpt]
\caption{Spirit types table (Chess)}
\begin{tabular}{ || l | l | l | l | l | l || }
\hline
Name            & Size &Move&Bonus&Spell&Flags\\
                &      &rate&     &Pts  &\\
\hline
pawn            &   30 & 0.5&   0 &   5 &\\
knight          &  120 & 1.0&  10 &  23 &H\\
bishop          &  150 & 2.0&   0 &  32 &F\\
rook            &  250 & 3.0&  30 &  48 &\\
queen           &  500 & 4.0&  50 &  74 &\\
king            & 1000 & 0.4&   0 &  83 &\\
master          & 2000 & 1.0&  10 &  157 &\\
grandmaster     & 3000 & 1.3&  20 &  217 &w\\
\hline
\end{tabular}
\end{table}
\end{same}
\begin{same}
\begin{table}[hbpt]
\caption{Spirit types table (Demonology)}
\begin{tabular}{ || l | l | l | l | l | l || }
\hline
Name            & Size &Move&Bonus&Spell&Flags\\
                &      &rate&     &Pts  &\\
\hline
imp             &   30 & 2.0&   0 &   9 &\\
lesser_demon    &   70 & 1.7&   0 &  15 &\\
hellhound       &  150 & 1.7&   0 &  25 &\\
tormented_soul  &  300 & 1.2& 100 &  37 &k\\
devil           &  250 & 1.0&   0 &  31 &\\
major_demon     &  500 & 2.0&   0 &  69 &F\\
balrog          & 1500 & 1.5&  50 &  156 &\\
demon_lord      & 2500 & 2.0&   0 &  200 &w\\
\hline
\end{tabular}
\end{table}
\end{same}
\begin{same}
\begin{table}[hbpt]
\caption{Spirit types table (Diana)}
\begin{tabular}{ || l | l | l | l | l | l || }
\hline
Name            & Size &Move&Bonus&Spell&Flags\\
                &      &rate&     &Pts  &\\
\hline
wolf            &   30 & 1.5&   0 &   8 &\\
swarm           &   50 & 1.3& 300 &  14 &Fk\\
mole            &   70 & 1.0&   0 &  15 &UL\\
snake           &  100 & 0.8&   0 &  15 &\\
shark           &  100 & 1.5&  10 &  17 &W\\
hawk            &  200 & 2.0&   0 &  38 &F\\
bear            &  300 & 1.4&   0 &  38 &\\
lion            &  500 & 1.5&   0 &  56 &L\\
terrasque       & 2000 & 0.5&   0 &  143 &L\\
\hline
\end{tabular}
\end{table}
\end{same}
\begin{same}
\begin{table}[hbpt]
\caption{Spirit types table (Inferno)}
\begin{tabular}{ || l | l | l | l | l | l || }
\hline
Name            & Size &Move&Bonus&Spell&Flags\\
                &      &rate&     &Pts  &\\
\hline
efreet          &   70 & 1.5&   0 &  12 &L\\
phoenix         &   30 & 2.0&   0 &  14 &FL\\
fire_giant      &  250 & 1.0&   0 &  31 &L\\
fire_drake      &  500 & 1.5&   0 &  56 &\\
lava_beast      & 1000 & 1.0&   0 &  98 &U\\
fire_elemental  & 2000 & 2.0&   0 &  162 &L\\
conflagration   & 3000 & 1.5&   0 &  210 &L\\
\hline
\end{tabular}
\end{table}
\end{same}
\begin{same}
\begin{table}[hbpt]
\caption{Spirit types table (Insects)}
\begin{tabular}{ || l | l | l | l | l | l || }
\hline
Name            & Size &Move&Bonus&Spell&Flags\\
                &      &rate&     &Pts  &\\
\hline
ant             &   30 & 1.0&   0 &   9 &UL\\
swarm           &   50 & 1.3& 300 &  14 &Fk\\
fly             &   70 & 1.0&   0 &  16 &F\\
moth            &  150 & 1.5&   0 &  29 &F\\
bee             &  250 & 1.2&  30 &  42 &F\\
grasshopper     &  350 & 1.0&   0 &  47 &HL\\
mosquito        &  500 & 0.5&   0 &  67 &FV\\
infestation     &  600 & 0.0&  50 &  68 &\\
roach           & 1000 & 0.8&  20 &  96 &\\
lobster         & 1500 & 1.0&   0 &  120 &W\\
creeping_doom   & 2500 & 0.8&  20 &  192 &\\
\hline
\end{tabular}
\end{table}
\end{same}
\begin{same}
\begin{table}[hbpt]
\caption{Spirit types table (Monsters)}
\begin{tabular}{ || l | l | l | l | l | l || }
\hline
Name            & Size &Move&Bonus&Spell&Flags\\
                &      &rate&     &Pts  &\\
\hline
spider          &   30 & 0.8&   0 &   5 &\\
yeti            &   70 & 1.0&   0 &  10 &L\\
ettin           &  120 & 1.5&   0 &  16 &\\
cyclops         &  150 & 0.8&   0 &  17 &\\
hydra           &  250 & 1.2&   0 &  26 &\\
crimson_death   &  500 & 2.0&   0 &  55 &F\\
sea_dragon      & 1000 & 1.2&   0 &  72 &W\\
green_dragon    & 1500 & 2.0&   0 &  105 &L\\
red_dragon      & 2500 & 1.5&   0 &  159 &F\\
gold_dragon     & 3500 & 2.0&   0 &  216 &Fw\\
\hline
\end{tabular}
\end{table}
\end{same}
\begin{same}
\begin{table}[hbpt]
\caption{Spirit types table (Necromancy)}
\begin{tabular}{ || l | l | l | l | l | l || }
\hline
Name            & Size &Move&Bonus&Spell&Flags\\
                &      &rate&     &Pts  &\\
\hline
ghost_ship      &    1 & 2.5&   0 &   7 &HcI\\
skeleton        &   40 & 0.8&   0 &   7 &\\
wraith          &   30 & 1.5&   0 &  10 &H\\
zombie          &   70 & 0.8&   0 &  15 &V\\
poltergeist     &  150 & 0.1&   0 &  18 &d\\
mummy           &   80 & 0.8&  30 &  18 &U\\
ghost           &  250 & 0.7&  20 &  40 &F\\
lacedon         &  500 & 1.0&   0 &  53 &W\\
lich            & 2000 & 2.0& -20 &  156 &w\\
vampire         & 1500 & 0.8&  30 &  149 &V\\
\hline
\end{tabular}
\end{table}
\end{same}
\begin{same}
\begin{table}[hbpt]
\caption{Spirit types table (Neptune)}
\begin{tabular}{ || l | l | l | l | l | l || }
\hline
Name            & Size &Move&Bonus&Spell&Flags\\
                &      &rate&     &Pts  &\\
\hline
living_ship     &    4 & 2.5&   0 &  10 &cI\\
pirana          &   30 & 1.2&   0 &   7 &W\\
water_nymph     &   50 & 1.2&   0 &  10 &W\\
whale           &   80 & 1.0&   0 &  13 &W\\
shark           &  100 & 1.5&  10 &  17 &W\\
sea_serpent     &  150 & 1.2&   0 &  27 &WL\\
craken          &  250 & 1.0&   0 &  31 &W\\
sea_giant       &  500 & 0.8&   0 &  51 &W\\
water_elemental & 2000 & 2.0&   0 &  162 &W\\
leviathan       & 3000 & 0.9&  10 &  200 &W\\
\hline
\end{tabular}
\end{table}
\end{same}
\begin{same}
\begin{table}[hbpt]
\caption{Spirit types table (Time)}
\begin{tabular}{ || l | l | l | l | l | l || }
\hline
Name            & Size &Move&Bonus&Spell&Flags\\
                &      &rate&     &Pts  &\\
\hline
diplodocus      &   30 & 1.3&   0 &   7 &\\
brontosaurus    &   70 & 0.7&   0 &  11 &\\
pleisiosaurus   &   50 & 1.5&   0 &  11 &W\\
pterodactyl     &  150 & 1.5&   0 &  29 &F\\
stegosaurus     &  250 & 1.0&   0 &  31 &\\
triceratops     &  500 & 0.8&   0 &  51 &L\\
tyrannosaurus   & 1500 & 1.5&   0 &  126 &\\
brachiosaurus   & 3000 & 0.5&   0 &  210 &WL\\
\hline
\end{tabular}
\end{table}
\end{same}
\begin{same}
\begin{table}[hbpt]
\caption{Spirit types table (Unity)}
\begin{tabular}{ || l | l | l | l | l | l || }
\hline
Name            & Size &Move&Bonus&Spell&Flags\\
                &      &rate&     &Pts  &\\
\hline
naga            &   30 & 1.0&   0 &   6 &\\
centaur         &   70 & 2.0&   0 &  16 &\\
werewolf        &  100 & 1.5&   0 &  19 &Vd\\
minotaur        &  150 & 1.0&   0 &  21 &L\\
owl_bear        &  250 & 1.0&   0 &  31 &\\
gryphon         &  350 & 2.0&   0 &  53 &F\\
sea_lion        &  600 & 1.5&   0 &  64 &W\\
chimera         & 1000 & 1.5&   0 &  105 &F\\
sphynx          & 2000 & 0.8&   0 &  172 &Fw\\
\hline
\end{tabular}
\end{table}
\end{same}
\begin{same}
\begin{table}[hbpt]
\caption{Spirit types table (Yavanna)}
\begin{tabular}{ || l | l | l | l | l | l || }
\hline
Name            & Size &Move&Bonus&Spell&Flags\\
                &      &rate&     &Pts  &\\
\hline
dryad           &   30 & 1.0&   0 &   6 &\\
tree_spirit     &   50 & 1.5&   0 &  14 &F\\
wood_beast      &   70 & 1.0&   0 &  12 &\\
magic_trees     &  150 & 1.0&   0 &  21 &\\
yellow_musk     &  200 & 0.7&   0 &  31 &V\\
ent             &  250 & 0.8&   0 &  30 &\\
swamp_beast     &  500 & 1.0&  10 &  55 &W\\
shambling_mound & 1200 & 0.8&  10 &  94 &\\
forest          & 2000 & 0.7&   0 &  146 &\\
\hline
\end{tabular}
\end{table}
\end{same}

\end{document}
